\documentclass[12pt]{article}
\usepackage[margin=0.4in]{geometry} 
\usepackage{polski}
\usepackage[polish]{babel}
\usepackage[utf8]{inputenc}
\pagenumbering{gobble}
\usepackage{amsmath, amsthm, amssymb, amsfonts, esint}
\usepackage{multicol}
\usepackage{xcolor}
\setlength{\columnseprule}{1pt}
\def\columnseprulecolor{\color{lightgray}}
\DeclareMathOperator{\der}{\operatorname{d}\!}
\newenvironment{bottompar}{\par\vspace*{\fill}}{\clearpage}
\newcommand{\grayrule}{{\color{lightgray} \hrule}}
\definecolor{m-gray}{gray}{0.36}
%%%%%%%%%%%%%%%%%%%%%%%%%%%%%%%%%%%%%%%%%%%%%%%%%%%%%%%%%%%%%%%%%%%%%%%%%%%%%%%%%

%%%%%%%%%%%%%%%%%%%%%%%%%%%%%%%%%%%%%%
%Do not alter this block.
\begin{document}
%%%%%%%%%%%%%%%%%%%%%%%%%%%%%%%%%%%%%%

\begin{multicols}{3}

% jednostki
\begin{multicols}{2}

\begin{equation*}
    \begin{split}
        \left[\Vec{E}\right] &= \frac{V}{m} \\
        \left[\Vec{H}\right] &= \frac{A}{m} \\
        \left[\Vec{D}\right] &= \frac{C}{m^2} \\
        \left[\Vec{B}\right] &= \frac{Vs}{m^2}
    \end{split}
\end{equation*}

\begin{equation*}
    \begin{split}
        \left[\Vec{J}\right] &= \frac{A}{m^2} \\
        \left[\overline{\overline{\varepsilon}}\right] &= \frac{F}{m} \\
        \left[\overline{\overline{\mu}}\right] &= \frac{H}{m} \\
        \left[\overline{\overline{\sigma}}\right] &= \frac{S}{m}
    \end{split}
\end{equation*}

\end{multicols}

% stałe w próżni
\begin{equation*}
    \begin{split}
        \varepsilon_0 &= \frac{10^{-9}}{36 \pi} \, \frac{F}{m} \\
        \mu_0 &= 4 \pi \cdot 10^{-7} \, \frac{H}{m} \\
        \rho_0 &= 0 \\
        j_0 &= 0
    \end{split}
\end{equation*}

\grayrule

% długość fali i częstotliwość
\begin{equation*}
    \lambda = \frac{c}{f}
\end{equation*}

\begin{equation*}
    f = \frac{1}{T}
\end{equation*}

\grayrule

% indukcje i pola
\begin{equation*}
    \Vec{D} = \overline{\overline{\varepsilon}} \cdot \Vec{E}
\end{equation*}

\begin{equation*}
    \Vec{B} = \overline{\overline{\mu}} \cdot \Vec{H}
\end{equation*}

\begin{equation*}
    \Vec{J} = \overline{\overline{\sigma}} \cdot \Vec{E}
\end{equation*}

\grayrule

% mnożenie wektorów

\begin{equation*}
    \Vec{a} \cdot \Vec{b} = a_x b_x + a_y b_y + a_z b_z
\end{equation*}

\begin{equation*}
    \Vec{a} \times \Vec{b}
        = \begin{vmatrix}
        \Vec{i_x} & \Vec{i_y} & \Vec{i_z} \\
        a_x & a_y & a_z \\
        b_x & b_y & b_z
        \end{vmatrix}
\end{equation*}

% operatory na skalarach i polach wektorowych
\begin{equation*}
    \operatorname{grad} f = \nabla f = \left(
            \frac{\partial f}{\partial x},\,
            \frac{\partial f}{\partial y},\,
            \frac{\partial f}{\partial z}
        \right)
\end{equation*}

\begin{equation*}
    \operatorname{div} \Vec{F}
        = \nabla \cdot \Vec{F}
        = \frac{\partial F_x}{\partial x}
            + \frac{\partial F_y}{\partial y}
            + \frac{\partial F_z}{\partial z}
\end{equation*}

\begin{equation*}
    \mathrm{rot}(\Vec{F}) = \nabla \times \Vec{F}
        = \begin{vmatrix}
        \Vec{i_x} & \Vec{i_y} & \Vec{i_z} \\
        \frac{\partial}{\partial x} & \frac{\partial}{\partial y} & \frac{\partial}{\partial z} \\
        F_x & F_y & F_z
        \end{vmatrix}
\end{equation*}

\grayrule
%%%%%%%%%%%%%%%%%%%%%%%%%%%%%%%%%%%%%%%%%%%%%%%%%%%%%%%%%%%%%%%%%%%%%%%%%%%%%
{ \color{m-gray}

    \begin{equation*}
        \oint\limits_C \Vec{F}\,\der \Vec{l}
            = \iint\limits_{S(C)} \nabla \times \Vec{F}\,\der\Vec{a} 
    \end{equation*}
    
    \begin{equation*}
        \oiint\limits_S \Vec{F} \, \der \Vec{a} = \iiint\limits_{V(S)} \nabla \Vec{F} \, \der\tau
    \end{equation*}
    
    \grayrule
    
    \begin{equation*}
        \begin{split}
            \oint\limits_C \Vec{E} \, \der\Vec{l} &= - \frac{\der\varphi}{\der t}\\
            \oint\limits_C \Vec{E} \, \der\Vec{l}
                &= - \frac{\der}{\der t} \int\limits_S \Vec{B} \, \der\Vec{s}
        \end{split}
    \end{equation*}
    
    \begin{equation*}
        \begin{split}
            \oint\limits_C \Vec{B} \, \der\Vec{l}
                &= \mu I + \mu \varepsilon \frac{\der\Phi_E}{\der t}\\
            \oint\limits_C \Vec{B} \, \der\Vec{l}
                &= \mu I + \mu \varepsilon \frac{\der}{\der t} \int\limits_S \Vec{E} \, \der\Vec{s}
        \end{split}
    \end{equation*}
    
    \begin{equation*}
        \varepsilon \, \oint\limits_S \Vec{E} \, \der\Vec{s} = q
    \end{equation*}
    
    \begin{equation*}
        \oint\limits_S \Vec{B} \, \der\Vec{s} = 0
    \end{equation*}

}
%%%%%%%%%%%%%%%%%%%%%%%%%%%%%%%%%%%%%%%%%%%%%%%%%%%%%%%%%%%%%%%%%%%%%%%%%%%%%
\grayrule

\begin{equation*}
    \nabla \times \Vec{E} = - \frac{\partial \Vec{B}}{\partial t}
\end{equation*}

\begin{equation*}
    \nabla \times \Vec{B} = \mu \Vec{j} + \mu \varepsilon \frac{\partial \Vec{E}}{\partial t}
\end{equation*}

\begin{equation*}
    \nabla \times \Vec{H} = \Vec{J} + \frac{\partial \Vec{D}}{\partial t}
\end{equation*}

\begin{equation*}
    \varepsilon \nabla \cdot \Vec{E} = \rho
\end{equation*}

\begin{equation*}
    \nabla \cdot \Vec{D} = \rho
\end{equation*}

\begin{equation*}
    \nabla \cdot \Vec{B} = 0
\end{equation*}

\grayrule

\begin{equation*}
    \Vec{J_D} = \frac{\partial \Vec{D}}{\partial t}
        = \varepsilon \frac{\partial \Vec{E}}{\partial t}
\end{equation*}

\grayrule

\begin{equation*}
    \frac{\partial \Vec{E}}{\partial t}
        = \frac{\partial (E \cdot e^{j \omega t})}{\partial t}
        = j \omega \cdot E \cdot e^{j \omega t}
\end{equation*}

\grayrule

\begin{equation*}
    \begin{split}
        \Vec{n} (\Vec{D_2} - \Vec{D_1}) &= \rho s \\
        \Vec{n} (\Vec{B_2} - \Vec{B_1}) &= 0 \\
        \Vec{n} \times (\Vec{E_2} - \Vec{E_1}) &= 0 \\
        \Vec{n} \times (\Vec{H_2} - \Vec{H_1}) &= \Vec{J_s}
    \end{split}
\end{equation*}

\end{multicols}
%%%%%%%%%%%%%%%%%%%%%%%%%%%%%%%%%%%%%%%%%%%%%%%%%%%%%%%%%%%%%%%%%%%%%%%%%%%%%

\begin{bottompar}
    {\footnotesize \ttfamily (CC BY-SA 4.0) 2019 Błażej Sewera \par
    src: https://github.com/jazzsewera/pofa-wzory}
\end{bottompar}

%%%%%%%%%%%%%%%%%%%%%%%%%%%%%%%%%%%%%%%%
%Do not alter anything below this line.
\end{document}
